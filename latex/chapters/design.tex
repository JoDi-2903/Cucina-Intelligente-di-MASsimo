%!TEX root = ../main.tex

\chapter{Design}
\label{chap:design}

\section{Agenten}
\label{sec:agenten}

Im Modell werden drei verschiedene Agenten eingesetzt: \emph{Customer Agents}, \emph{Service Agents} sowie ein \emph{Manager Agent}. Die Agenten sind in einem Multi-Agenten-System organisiert, in dem sie miteinander kommunizieren und sich gegenseitig beeinflussen.

\subsection{Customer Agents}
\label{subsec:customer_agents}
Customer Agents beschreiben eine Gruppe von Kunden, die ein Restaurant besuchen und eine Bestellung aufgeben wollen. Die Attribute der Agenten werden größtenteils randomisiert generiert. Dazu zählen die genaue Anzahl der Personen, die maximale Wartezeit sowie das ausgewählte Gericht.

Die Agenten werden über einen internen Status gesteuert, der den aktuellen Zustand des Agenten beschreibt. Die möglichen Zustände sind \emph{Waiting for Service Agent}, \emph{Waiting for food}, \emph{Eating}, \emph{Finished Eating}, \emph{Rejected} und \emph{Done}. Der Status wird durch die verschiedenen Aktionen des Agenten verändert. 

Wichtig im Zuge dieses Konzepts ist, dass die Agenten das Restaurant nicht verlassen, sobald ihre Zeit abgelaufen ist. Stattdessen wird das Essen beendet, dafür sinkt jedoch die Bewertung.

Wenn die Customer Agents ihr Essen beendet haben, geben sie eine Bewertung ab. Diese Bewertung ist eine Funktion mit mehreren Zufallsfaktoren:

\begin{equation*}
    r = \text{round}\left(\max\left(r_{\text{min}}, \min\left(r_{\text{max}}, r_{\text{max}} - \alpha \cdot \text{exceedance} - \beta \cdot \text{error} + \gamma\right)\right), 2\right)
\end{equation*}

wobei:
\begin{itemize}
    \item $r$ die finale Bewertung ist,
    \item $r_{\text{min}}$ die minimal mögliche Bewertung ist,
    \item $r_{\text{max}}$ die maximal mögliche Bewertung ist,
    \item $\alpha$ die Gewichtung der Wartezeitstrafe ist,
    \item $exceedance$ der Quotient aus der tatsächlichen Wartezeit und der angegebenen maximalen Zeit ist (nur bei Überschreitung, sonst 0),
    \item $\beta$ die Gewichtung für fehlerhafte Bestellungen ist,
    \item $error$ eine zufällige Fehlerquote ist,
    \item $\gamma$ eine zufällige Bewertungsvariabilität abhängig von der Gruppengröße ist.
\end{itemize}

Wird ein Kunde abgewiesen, gibt er immer die niedrigste Bewertung ab.

Im \emph{PEAS-Framework} lassen sich die Customer Agents wie folgt beschreiben:
\begin{description}
    \item[Performance Measure] Einhaltung der maximalen Zeit
    \item[Environment] Das Restaurant, Service Agents
    \item[Actuators] Bestellung aufgeben, Essen bewerten
    \item[Sensors] Wartezeit, interner Status
\end{description}

Bei den Customer Agents handelt es sich gemäß des \emph{AIMA-Framework}s um \emph{Simple Reflex Agents}, da sie ihre Aktionen basierend auf dem aktuellen Zustand und den wahrgenommenen Informationen ausführen, ohne eine interne Zustandsrepräsentation der Welt zu verwenden. Sie reagieren direkt auf die wahrgenommenen Reize, wie z.B. die erhaltene Bestellung, um ihre nächsten Aktionen zu bestimmen.


\subsection{Service Agents}
\label{subsec:service_agents}
Service Agents sind für die Bedienung der Customer Agents zuständig. Sie haben die Aufgabe, die Bestellungen der Kunden entgegenzunehmen, diese zuzubereiten und das Essen zu servieren. Jeder Service Agent hat eine Kapazität an Customer Agents, die er bedienen kann. Einmal zugewiesen, werden die Customer Agents nicht mehr von anderen Service Agents bedient, außer der Service Agent beendet seine Arbeit, dann werden seine Customer Agents an die übrigen Service Agents übergeben.

Jeder Service Agent kann pro Zeitschritt nur zwei Customer Agents bedienen, dabei wird ein Customer Agent entweder platziert oder abgewiesen und in die Status \emph{Waiting for food} bzw \emph{Rejected} versetzt, während für einen anderen Customer Agent die Zubereitung der Speisen durchgeführt wird. Da die Zubereitung der Speisen im Menü festgelegt ist und unterschiedlich lange dauert, wird deshalb entweder die Zubereitungszeit um einen Zeitschritt reduziert oder, sofern die Zubereitung abgeschlossen ist, das Essen serviert und der Customer Agent in den Status \emph{Eating} versetzt.\\
Kunden werden dann abgewiesen, wenn sie im Status \emph{Waiting for Service Agent} sind, aber die Summe aus Zubereitungs- und Essenszeit der Speisen die maximale Wartezeit überschreitet, also ein Service in der angegeben Zeit überhaupt nicht möglich ist.

Die zu bedienenden Customer Agents werden von den Service Agents in jedem Schritt dynamisch sortiert und entsprechend bedient. Dabei werden verschiedene Faktoren berücksichtigt und gewichtet, um die Interessen der Kunden (also die Minimierung der Wartezeit) sowie die Interessen des Managers (also die Maximierung des Gewinns) zu berücksichtigen.\\
Dabei wird für jeden Kunden ein Rang bestimmt - je höher der Rang, desto früher wird der Kunde bedient. Die Sortierung für neue Kunden erfolgt wie folgt:
\begin{equation*}
    r = \alpha \cdot p \cdot n + \beta \cdot \text{total\_time} + \gamma \cdot \text{time\_left}
\end{equation*}

wobei:
\begin{itemize}
    \item $r$ der Rang des Kunden ist,
    \item $p$ der Preis des Gerichts ist,
    \item $n$ die Anzahl der Personen ist,
    \item $\alpha$ die Gewichtung des Profits ist,
    \item $total\_time$ die Gesamtzeit ist, die der Kunde bereits im Restaurant verbracht hat,
    \item $\beta$ die Gewichtung der Gesamtzeit ist,
    \item $time\_left$ die verbleibende Zeit ist,
    \item $\gamma$ die Gewichtung der verbleibenden Zeit ist
\end{itemize}

Für die Sortierung der bereits im Restaurant befindlichen Kunden wird die gleiche Formel verwendet, zusätzlich jedoch noch die Zubereitungszeit der Speisen berücksichtigt wird. Damit ergibt sich folgende Formel:
\begin{equation*}
    r = \alpha \cdot p \cdot n + \beta \cdot \text{total\_time} + \gamma \cdot \text{time\_left} + \delta \cdot \text{cooking\_time}
\end{equation*}

wobei die Werte entsprechend der Formel für neue Kunden definiert sind, $cooking\_time$ die Zubereitungszeit des Gerichts und $\delta$ die Gewichtung der Zubereitungszeit ist.

Im \emph{PEAS-Framework} lassen sich die Service Agents wie folgt beschreiben:
\begin{description}
    \item[Performance Measure] Wartezeit der Kunden minimieren, möglichst hohe Bewertung, Maximierung des Gewinns
    \item[Environment] Das Restaurant, Customer Agents
    \item[Actuators] Bestellung aufnehmen, Essen zubereiten, Essen servieren, Kunden abweisen
    \item[Sensors] Neue Kunden vorhanden, Zustand der Customer Agents
\end{description}

Bei den Service Agents handelt es sich gemäß des \emph{AIMA-Framework}s ebenfalls um \emph{Simple Reflex Agents}, da sie ihre Aktionen basierend auf dem aktuellen Zustand und den wahrgenommenen Informationen ausführen, ohne eine interne Zustandsrepräsentation der Welt zu verwenden. Sie reagieren direkt auf die wahrgenommenen Reize, wie z.B. den Status der Kunden.


\subsection{Manager Agent}
\label{subsec:manager_agent}
Der Manager Agent ist für die Koordination der Service Agents zuständig. Er existiert im Restaurant nur ein Mal und hat die Aufgabe, die Service Agents zu verwalten.